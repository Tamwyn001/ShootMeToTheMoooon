\documentclass[
    oneside, 
    footinclude=off, 
    captions=tableheading, 
    DIV=12;usenames,
    dvipsnames
]{scrartcl}

% -----------------------------------------------------------
% Pakete
\usepackage{UniLaTeXPackage}

% behilfsmäßige Definition, wird noch in das uni package verlagert
\newenvironment{ergebnis}{\begin{tcolorbox}\textbf{Ergebnis}}{\end{tcolorbox}}

\begin{document}

% -----------------------------------------------------------
% Deckblatteinrichtung
    \title{Shoot me to the moon}
    \subtitle{Computerphysik 1, Abschlussprojekt (Universität Konstanz)}
    \author{Autoren: Arto Steffan, Tom Folgmann, David Jannack \\ \large{Tutor: Daniel Kazenwadel und Jakob }}
    \date{Abgabe am 01.10.2023}
    \maketitle
    \pagenumbering{alph}
    \thispagestyle{empty}
    \section*{Einleitung}
        \subfile{Sektionen/0-Einleitung.tex}

    \newpage

% -----------------------------------------------------------
% Inhaltsvorbereitung

    \tableofcontents
    \thispagestyle{empty}	
    \newpage
    \setcounter{page}{1}
    \pagenumbering{arabic}
% -----------------------------------------------------------
% Link zum Code
Das vollständig Code des Projektes befindet sich unter dem Git Repostitory: \url{https://github.com/Tamwyn001/ShootMeToTheMoooon}.

% -----------------------------------------------------------
% Dokumentkern

\newpage
\section{Grundlagen}
    \subfile{Sektionen/1-Grundlagenteil.tex}
	

\newpage
\section{Code}
    \subfile{Sektionen/2-Code.tex}

\newpage
\section{Auswertung}
    \subfile{Sektionen/3-Auswertung.tex}

\newpage
\section{Fazit}
    \subfile{Sektionen/4-Fazit.tex}

% -----------------------------------------------------------
% Abspann und Credits

\newpage
    \subfile{Sektionen/5-Literatur.tex}
\newpage
    \listoffigures
    \listoftables


%\includepdf[pages=1-1,scale=0.9,frame=true]{Protokolle/PATH.pdf}
\end{document}
