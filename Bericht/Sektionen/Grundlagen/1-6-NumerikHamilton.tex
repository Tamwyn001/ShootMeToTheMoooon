\documentclass{subfiles}

\begin{document}
    Um die Theorie der Numerik auf unser Problem anwenden zu können, müssen wir noch eine \emph{rechte Seite} definieren. Wir finden diese über die \emph{Hamiltonischen Bewegungsgleichungen}, welche durch die Richtungsableitungen in $q(t)$ und $p(t)$ gegeben und in einem Tupel zusammengefasst werden. Hierzu bestimmen wir zunächst
    \begin{align*}
        \dv{a}H(t,(a,p(t))) &= \dv{a}\Biggl(
                T + V(t,a)
        \Biggr) = \dv{a}V(t,a),\numberthis\label{eq:dHa}\tag{dHa}\\
        \dv{b}H(t,(q(t),b)) &= \dv{b}\Biggl(
                \frac{b^2}{2m_S} + V(t,q(t))
        \Biggr) = \frac{b}{m_S} + \ubra{\dv{b}V(t,q(t))}{=0}.\numberthis\label{eq:dHb}\tag{dHb}
    \end{align*}    
    An dieser Stelle müssen wir noch einmal arbeiten, um die Ableitung des Potentials in $q(t)$ zu bestimmen, da es sich um eine Verkettung höheren Grades handelt. Definiere
    \begin{align*}
        f &:= \nbra{\begin{pmatrix}s_1\cdot \sin(s_2)\cdot\cos(s_3) - a\cdot\cos(\Phi_M(t))\\ s_1\cdot\sin(s_2)\cdot\sin(s_3) - b\cdot\sin(\Phi_M(t))\\ 0\end{pmatrix}}_{s\in\R^3},\\
        g &:= \nbra{\dabs{s}{2}\bigr}_{(t,s)\in\R\times\R^3},
    \end{align*}
    sodass $V(t,q(t)) = G\cdot m_1\cdot m_S\cdot q_1^*(t)^{-1} + G\cdot m_2\cdot m_S\cdot g(f(t,q(t)))^{-1}$. Wir berechnen folgens die partiellen Ableitungen der Potentialfunktion.
    \begin{align*}
        \dv{s}V\left.\bigl(t,(q(t) + s\cdot\mbbEins_1)\bigr)\right|_{s=0} &= -G\cdot m_1\cdot m_S\cdot q_1^*(t)^{-2} + G\cdot m_2\cdot m_S\cdot \dv{a}g(f(t,q(t) + a\cdot\mbbEins_1))^{-1}\vert_{a=0}.
    \end{align*}
    Der zweite Summand berechnet sich mit der Kettenregel. Wir definieren vorab die Abkürzung $\mbbEins_{0,k}:=(0,\mbbEins_k)$ für $k\in[3]$:
    \begin{align*}
        \dv{a}g(f(q(t) + a\cdot\mbbEins_{0,1}))\vert_{a=0} &= d(\inv \circ g\circ f)(q(t))(\mbbEins_1) = d\inv(g(f(q(t))))(d(g\circ f)(q(t)(\mbbEins_{0,1})))\\
        &= \inv((g(f(q(t))))^2)\cdot d(g\circ f)(q(t))(\mbbEins_{0,1})\\
        &= \frac{1}{g(f(q(t)))}\cdot d(g\circ f)(q(t))(\mbbEins_{0,1}).
    \end{align*}
    Erneute Anwendung der Kettenregel auf den zweiten Faktor liefert
    \begin{align}
        d(g\circ f)(q(t))(\mbbEins_{0,1}) &= dg(f(q(t)))(df(q(t))(\mbbEins_{0,1})).
        \label{eq:NumerikHamiltonPotentialZweiteVerkettung}
    \end{align}
    Es verbleibt noch die Richtungsableitungen der Quadratnorm $g$, sowie $f$ zu berechnen:
    \begin{itemize}
        \item Schreibe explizit $g(x)=\left(\sum_{j\in[3]}x_j^2\right)^{1/2}$ für $x\in\R^3$, dann (mit der Richtung $h\in\R^3$ und $q_2$ bezeichnend für das Quadratmonom)
        \begin{align*}
            dg(x)(h) &= \dv{a}\left.\left(\sum_{j\in[3]}(x_j + a\cdot h_j)^2\right)^{1/2}\right\vert_{a=0}\\
            &= \dv{a}\sum_{j\in[3]}(x_j + a\cdot h_j)\right\vert_{a=0}\cdot\frac{1}{2}\cdot\left(\sum_{j\in[3]}x_j^2\right)^{-1/2}\\
            &=\left(\sum_{j\in[3]}dq_2(x_j)(h_j)\right)\cdot\frac{1}{2}\cdot\frac{1}{g(x)}
            =\frac{\sum_{j\in[3]}h_j\cdot x_j}{g(x)}.
        \end{align*}

        \item Die erste partielle Ableitung von $f$ ergibt sich als
        \begin{align*}
            df(q(t))(\mbbEins_{0,1}) &= \dv{s_1}\left.\nbra{\begin{pmatrix}s_1\cdot \sin(q_2^*(t))\cdot\cos(q_3^*(t)) - a\cdot\cos(\Phi_M(t))\\ s_1\cdot\sin(q_2^*(t))\cdot\sin(q_3^*(t)) - b\cdot\sin(\Phi_M(t))\\ 0\end{pmatrix}}_{s\in\R^3}\right\vert_{s_1=q_1^*(t)}\\
            &= \begin{pmatrix}\sin(q_2^*(t))\cdot\cos(q_3^*(t)) \\ \sin(q_2^*(t))\cdot\sin(q_3^*(t)) \\ 0\end{pmatrix}
        \end{align*}
    \end{itemize}
    Rückeinsetzen in \eqref{eq:NumerikHamiltonPotentialZweiteVerkettung} ergibt 
    \begin{align*}
        d(g\circ f)(q(t))(\mbbEins_1) &= \frac{q_1^*(t)\cdot \sin(q_2^*(t))\cdot\cos(q_3^*(t)) - a\cdot\cos(\Phi_M(t))}{g(f(q(t)))}\cdot\sin(q_2^*(t))\cdot\cos(q_3^*(t))\\
        &+ \frac{x_2}{g(x)}\cdot \sin(q_2^*(t))\cdot\sin(q_3^*(t)).
    \end{align*}

    sodaß wir durch Kenntnis der Ableitung von inv auf $\R$, gegeben durch $d$inv$(x)(1) = ($inv$\circ q_2)(x)$ mit $q_2$ als Quadratmonom ferner schreiben können:
    \[
        \dv{a}\frac{1}{g_i(a)} = d(\inv\circ g_i)(a)(1) = (\inv\circ q_2\circ g_i)(a)\cdot dg_i(a)(1) = \frac{1}{g_i(a)^2}\cdot \dv{a}g_i(a).  
    \]
    % Dabei verstecken wir in $h_a = (0,(1,0))$ die verwendete Ableitungsrichtung gemäß der tatsächlichen Argumente von $H$. 
    Ferner ist $dg_i(x)(h)$ für $h\in\R^3$ zu evaluieren (wir erinnern uns, daß unser $a$ einen \emph{Platzhalter} für den später eingesetzten Wegpunkt $q(t)\in\R^3$ darstellt), was wir durch $l_{i,j}:=\bigl((Q_i(t)_j - x)^2\bigr)_{x\in\R}$ fortsetzen zu
    \[
        dg_i(a)(h) = \sum_{k\in[3]}h_k\cdot dg_i(a)(\mbbEins_k), \qquad\mbbEins_{k}:=\fdef{\begin{cases}
            1 & i = k\\
            0 & \text{sonst}
        \end{cases}}{i\in[3]}
    \]
    sodaß die Ableitung von $g$ für die Richtung $h=1\in\R^3$ in der $k$-ten Komponente der obigen Summe zusammen mit der Quadratnorm $\dabs{x}{2} := \bigl(x_1^2 + x_2^2 + x_3^2\bigr)^{1/2}$ die Form 
    \[
        dg_i(a)(\mbbEins_k) = \dv{s}\left.\nbra{l_{i,k}(a_k + s) + \sum_{j\in[3]\setminus{\{k\}}}l_{i,j}(a_j)}^{\frac12}\right|_{s=0}
    \]
    besitzt, also beispielsweise für $k=1$ gerade
    \[
        dg_i(a)(\mbbEins_1) = \dv{a_1}\nbra{\ubra{(Q_i(t)_1 - a_1)^2}{l_{i,1}(a_1)} + \ubra{(Q_i(t)_2 - a_2)^2}{l_{i,2}(a_2)} + \ubra{(Q_i(t)_3 - a_3)^2}{l_{i,3}(a_3)}}^{\frac12}.
    \]
    Leiten wir diese erste Richtung speziell ab, so erhalten wir für $W_i(a) = \sum_{k\in[3]}l_{i,k}(a_k)$ als Zusammenfassung des Wurzelinhaltes
    \[
        \dv{s}\left.\sqrt{W_i(a + s\cdot\mbbEins_1)}\right|_{s=0} = \ubra{\frac{1}{\sqrt{W_i(a)}}}{=\dabs{Q_i(t) - a}{2}^{-1}}\cdot\dv{a_1}l_{i,1}(a_1) = \ubra{\frac{1}{\dabs{Q_i(t) - a}{2}}}{= 1/g_i(a)}\cdot 2\cdot (Q_i(t)_1 - a_1). 
    \]
    Es findet sich so durch die Symmetrie der symbolisch gesammelte Ableitungsausdruck $\dv{a_k}g_i(a) = 2\cdot (Q_i(t)_k - a_k) / g_i(a)$. Die Zusammenfassung derer ist wiederum der Ableitungstensor von $g_i$ in $a$:
    \[
        g_i'(a) = \dv{a}g_i(a) = \frac{2}{g_i(a)}\cdot\fdef{Q_i(t)_k - a_k}{k\in[3]},
    \]
    sodaß sich unsere anfangs gesuchte Ableitung des Potentials in $a$ zu 
    \[
        \dv{a}V(t,a) = G\cdot m_S\cdot\sum_{i\in[2]}m_i\cdot\frac{2}{g_i(a)^3}\cdot\fdef{Q_i(t)_k - a_k}{k\in[3]}\numberthis\label{eq:dV}\tag{dV}
    \]
    zusammensetzt. Damit erhalten wir die Hamiltonischen Bewegungsgleichungen und gleichzeitig durch die Defintion $F(t,u(t)):=\bigl(-\dv{b}H(t,(u_1^*(t),b))_{b=u_2^*(t)},\dv{a}H(t,(a,u_2^*(t)))|_{a=u_1^*(t)}\bigr)$ die rechte Seite des Hamiltonsystems:
    \begin{align*}
        F(t,u(t)) = \begin{pmatrix}
            -u_2^*(t)/m_S\\
            \dv{a}V(t,a)|_{a=u_1^* (t)}
        \end{pmatrix}, \qquad u(t) = \begin{pmatrix}
            q(t)\\
            p(t)
        \end{pmatrix}.
    \end{align*}

\end{document}