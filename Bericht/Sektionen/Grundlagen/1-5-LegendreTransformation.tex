\documentclass{subfiles}

\begin{document}
    Um die Hamiltonfunktion aufstellen zu können, braucht es eine Transformation der Form $p(t):=\dv{b}L(t,(q(t),b))|_{b=q'(t)}$.
    Wir müssen nach Form unserer kinetischen Energie die Ableitungen komponentenweise betrachten. Wir notieren zunächst den Ableitungsvektor
    \[
        \dv{b}L\left.\bigl(t,(q(t),b)\bigr)\right|_{b=q'(t)} = \fdef{
            \dv{s}L\left.\bigl(t,(q(t),q'(t) + s\cdot\mbbEins_k)\bigr)\right|_{s=0} \\
        }{k\in[3]}.
    \]
    Somit erhalten wir konkret:
    \begin{align*}
        \dv{s}L\left.\bigl(t,(q(t),q'(t) + s\cdot\mbbEins_1)\bigr)\right|_{s=0} &= \frac12\cdot m_S\cdot\dv{b_1}b_1^2|_{b_1 = q'(t)_1}\\
        &= m_S\cdot q'(t)_1, \\
        \dv{s}L\left.\bigl(t,(q(t),q'(t) + s\cdot\mbbEins_2)\bigr)\right|_{s=0} &= \frac12\cdot m_S\cdot q(t)_2^2\cdot\dv{b_2}b_2^2|_{b_2 = q'(t)_2} \\
        &= m_S\cdot q(t)_2\cdot q'(t)_2,\\
        \dv{s}L\left.\bigl(t,(q(t),q'(t) + s\cdot\mbbEins_3)\bigr)\right|_{s=0} &= \frac12\cdot m_S\cdot q(t)_3^2\cdot q(t)_1^2\cdot\sin(q(t)_2)^2\cdot\dv{b_3}b_3^2|_{b_3 = q'(t)_3} \\
        &= m_S\cdot q(t)_3\cdot q(t)_1^2\cdot\sin(q(t)_2)^2\cdot q'(t)_3.
    \end{align*}

    und Substitution $q'(t) = p(t) / m_S$ den Hamiltonian der Form 
    \[
        H(t,(q(t),p(t))) = \frac{p(t)^2}{2m_S} + V(t,q(t))
    \]
    erhalten.
\end{document}