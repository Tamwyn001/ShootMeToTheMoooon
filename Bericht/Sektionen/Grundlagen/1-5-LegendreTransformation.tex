\documentclass{subfiles}

\begin{document}
    Um die Hamiltonfunktion aufstellen zu können, braucht es eine Transformation der Form $p(t):=\dv{b}L(t,(q(t),b))|_{b=q'(t)}$.
    Wir müssen nach Form unserer kinetischen Energie die Ableitungen komponentenweise betrachten. Wir notieren zunächst den Ableitungsvektor
    \[
        \dv{b}L\left.\bigl(t,(q(t),b)\bigr)\right|_{b=q'(t)} = \fdef{
            \dv{s}L\left.\bigl(t,(q(t),q'(t) + s\cdot\mbbEins_k)\bigr)\right|_{s=0} \\
        }{k\in[3]},\quad \mbbEins_{k}:=\fdef{\begin{cases}
            1 & i = k\\
            0 & \text{sonst}
        \end{cases}}{i\in[3]}
    \]
    Somit erhalten wir konkret:
    \begin{align*}
        \dv{s}L\left.\bigl(t,(q(t),q'(t) + s\cdot\mbbEins_1)\bigr)\right|_{s=0} &= \frac12\cdot m_S\cdot\dv{b_1}b_1^2|_{b_1 = q'(t)_1}\\
        &= m_S\cdot q'(t)_1, \\
        \dv{s}L\left.\bigl(t,(q(t),q'(t) + s\cdot\mbbEins_2)\bigr)\right|_{s=0} &= \frac12\cdot m_S\cdot q(t)_1^2\cdot\dv{b_2}b_2^2|_{b_2 = q'(t)_2} \\
        &= m_S\cdot q(t)_1^2\cdot q'(t)_2,\\
        \dv{s}L\left.\bigl(t,(q(t),q'(t) + s\cdot\mbbEins_3)\bigr)\right|_{s=0} &= \frac12\cdot m_S\cdot q(t)_1^2\cdot\sin(q(t)_2)^2\cdot\dv{b_3}b_3^2|_{b_3 = q'(t)_3} \\
        &= m_S\cdot q(t)_1^2\cdot\sin(q(t)_2)^2\cdot q'(t)_3.
    \end{align*}
    Invertieren wir nun die einzelnen Komponenten, so können wir die Ableitung der verallgemeinerten Koordinaten mittels des Argumentes $\bigl(t,(q(t),p(t))\bigr)$ ausdrücken. Wir erhalten
    \[
        q'\bigl(t,(q(t),p(t))\bigr) = \frac{1}{m_S}\cdot\left(
            p(t)_1,
            \frac{p(t)_2}{q(t)_1^2},
            \frac{p(t)_3}{q(t)_1^2\cdot\sin(q(t)_2)^2}
        \right).
    \]
    Setzen wir dies für die Vorkommnisse von $q'(t)$ im Lagrangian $L$ ein, so erhalten wir den Hamolonian $H$ in der Form
    \[
        H\bigl(t,(q(t),p(t))\bigr) = T\bigl(t,(q(t),p(t))\bigr) + V\bigl(t,q(t)\bigr),
    \]
    wobei nach obiger Defintion für den kinetischen Energieanteil gilt
    \begin{align*}
        T(t,(q(t),p(t))) &= \frac{1}{2\cdot m_S}\cdot\nbra{
            p(t)_1^2 + \nbra{
                \frac{p(t)_2}{q(t)_1^2}
            }^2\cdot q(t)_1^2 + \nbra{
                \frac{p(t)_3}{q(t)_1^2\cdot\sin(q(t)_2)^2}
            }^2\cdot q(t)_1^2\cdot\sin(q(t)_2)^2
        } \\
        &= \frac{1}{2\cdot m_S}\cdot\left(
            p(t)_1^2 + \frac{p(t)_2^2}{q(t)_1^2} + \frac{p(t)_3^2}{q(t)_1^2\cdot\sin(q(t)_2)^2}
        \right).
    \end{align*}
    Für die Ableitung des Hamiltonians nach $p(t)$ müssen wir lediglich die Ableitung $\dv{b}T(t,(q(t),b))|_{b = p(t)}$ durchführen. Wir erhalten dabei drei Komponenten, wie auch schon bei der Legendre Transformation. Diese lauten hier
    \begin{align*}
        \dv{s}T\left.\bigl(t,(q(t),p(t) + s\cdot\mbbEins_1)\bigr)\right|_{s=0} &= \frac{1}{2\cdot m_S}\cdot\dv{b_1}b_1^2|_{b_1 = p(t)_1} \\
        &= \frac{1}{m_S}\cdot p(t)_1, \\
        \dv{s}T\left.\bigl(t,(q(t),p(t) + s\cdot\mbbEins_2)\bigr)\right|_{s=0} &= \frac{1}{2\cdot m_S\cdot q(t)_1^2}\cdot\dv{b_2}b_2^2|_{b_2 = p(t)_2} \\
        &= \frac{1}{m_S\cdot q(t)_1^2}\cdot p(t)_2, \\
        \dv{s}T\left.\bigl(t,(q(t),p(t) + s\cdot\mbbEins_3)\bigr)\right|_{s=0} &= \frac{1}{2\cdot m_S\cdot q(t)_1^2\cdot\sin(q(t)_2)^2}\cdot\dv{b_3}b_3^2|_{b_3 = p(t)_3} \\
        &= \frac{1}{m_S\cdot q(t)_1^2\cdot\sin(q(t)_2)^2}\cdot p(t)_3.
    \end{align*}
    Die Zusammenfassung bildet dann die nötige Richtungsableitung für die zweite Hamiltonsche Bewegungsgleichung. 
\end{document}