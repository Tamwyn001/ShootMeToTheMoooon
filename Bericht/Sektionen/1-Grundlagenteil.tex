\documentclass{subfiles}

\begin{document}
    Zunächst ist die Situation mathematisch zu beschreiben, wobei wir uns aufgrund der numerischen Bekanntheit für eine Hamilton-Modellierung entscheiden. Zur Erinnerung: Für ein Potential $V:\R\times\R^d\to\R$ hat die Hamiltonfunktion die Form 
    \[
        H(t,(q(t),p(t))) = \frac{p(t)^2}{2m} + V(t,q(t)).
    \]
    Um eine konkrete Abbildungsvorschrift zu erlangen, stellen wir die Lagrange Funktion des Systems auf und finden mithilfe von einer Potentialüberlegung und der Legendre Transformation die schließliche Form. 

    \subsection*{Grundlegende Überlegung}
        \subfile{Grundlagen/1-1-GrundlegendeUeberlegungen.tex}
        
    \subsection*{Potentialbeschreibung}
        \subfile{Grundlagen/1-2-Potentialbeschreibung.tex}

    \subsection*{Kinetische Energie in Kugelkoordinaten}
        \subfile{Grundlagen/1-3-KinetischeEnergie.tex}

    \subsection*{Lagrange Funktion}
        \subfile{Grundlagen/1-4-LagrangeFunktion.tex}

    \subsection*{Legendre Transformation}
        \subfile{Grundlagen/1-5-LegendreTransformation.tex}

    \subsection*{Numerik des Hamiltonsystems}
        \subfile{Grundlagen/1-6-NumerikHamilton.tex}
    
    \subsubsection*{Optionale Matrixdarstellung}
        \subfile{Grundlagen/1-7-Matrixdarstellung.tex}

\end{document}