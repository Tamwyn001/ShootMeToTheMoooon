\documentclass{subfiles}

\begin{document}
    

    Zunächst ist die Situation mathematisch zu beschreiben, wobei wir uns aufgrund der numerischen Bekanntheit für eine Hamilton-Modellierung entscheiden. Zur Erinnerung: Für ein Potential $V:\R\times\R^d\to\R$ hat die Hamiltonfunktion die Form 
    \[
        H(t,(q(t),p(t))) = \frac{p(t)^2}{2m} + V(t,q(t)).
    \]
    Um eine konkrete Abbildungsvorschrift zu erlangen, stellen wir die Lagrange Funktion des Systems auf und finden mithilfe von einer Potentialüberlegung und der Legendre Transformation die schließliche Form. 

    \subsection*{Grundlegende Überlegung}
        Wir konzentrieren uns auf ein Dreikörperproblem, bei welchem eine Masse (diejenige des Satelliten) im Vergleich zu den anderen beiden Körpermassen der Erde und des Mondes klein ist. Als Ortskoordinaten der drei Körper verwenden wir Wege der Form 
        \[
            q:\R\to\R^3,\qquad t\mapsto (x_1^*(t),x_2^*(t),x_3^*(t)),
        \]
        wobei $x:\R\to\R^3$ unser tatsächlicher Aufenthaltsort in kartesischen Koordinaten ist. Wir legen die Erde mit Zentrum im Ursprung fest und setzen ihren Radius auf $r_E:=6300\si{\kilo\metre}$, sodaß unser $x_0$ Element des Randes $\delta B_{r_E}:=\{x\in\R^3:\dabs{x}{3} = r_A\}$ sein wird. Ziel ist das Erreichen des Ortes
        \[
            x_1\in B_{r_M,d}:=(B_{r_M}+d):=\{x + d\in\R^3:\dabs{x}{2} = r_M\}
        \]
        festgelegt durch den Zentrumsvektor $d$ des Mondes zum Erdzentrum und den Mondradius $r_M:=??\si{\kilo\metre}$.

    \subsection*{Potentialbeschreibung}
        Für das \emph{Gravitationspotential} des Systems wählen wir zwei Newtonsche Potentiale an den jeweiligen Zentren des Mondes und der Erde. Allgemein schreiben wir:
        \[
            V:\R\times\R^3\to \R,\qquad (t,q(t),q'(t))\mapsto G\cdot\sum_{i\in[2]}\frac{m_i\cdot m_S}{\dabs{Q_i^*(t) - q(t)}{2}},
        \]
        wobei wir in $Q$ die Orte der Erde (an Stelle $1$) und des Mondes (an Stelle $2$) sammeln und mit $m_S$ die Satellitenmasse bzw. mit $m:=(m_E,m_M)$ die Erd- bzw. Mondmasse bezeichnen:
        \[
            Q:\R\to\R^3,\qquad t\mapsto \bigl(0_\R^3,d\bigr).
        \]
        Wir lassen hier bewusst die Masse des Satelliten selbst aus, da wir sonst eine Singularität an jeder Stelle durch $1/\dabs{q(t)-q(t)}{2}$ erhielten. 

    \subsection*{Lagrange Funktion}
        Der Lagrangeformalismus beschreibt den Unterschied der \emph{kinetischen} zur \emph{potentiellen Energie} des Systems. Als kinetische Energie betrachten wir allgemein die Summe der einzelnen kinetischen Energien der beteiligten Körper als
        \[
            T(t,(q(t),q'(t))) = \sum_{i\in[3]}\frac12\cdot m_i\cdot\dabs{(Q_i^*)'(t)}{2}^2 + \frac12\cdot m_S\cdot\dabs{q'(t)}{2}^2.
        \]
        % -> Sind die Q(t) bzw. q(t) hier richtig implementiert?
        Unter der Annahme ruhender Erde und Mondes können wir $T_0(t,(q(t),q'(t)))$ über $m_S\cdot \dabs{q'(t)}{2}^2 / 2$ beschreiben. Zusammen mit dem oben bereits bestimmten Potential $V$ erhalten wir durch die Form $L = T - V\circ\phi$ mit der Argumentauswahlfunktion $\phi:=\bigl((t,x_1)\bigr)_{(t,x)}$ den Lagrangian
        \[
            L:\R\times\bigl(\R^3\bigr)^2\to\R,\qquad \bigl(t,(q(t),q'(t))\bigr)\mapsto \frac12\cdot m_S\cdot\dabs{q'(t)}{^2} - G\cdot\sum_{i\in[2]}\frac{m_i\cdot m_S}{\dabs{Q_i^*(t) - q(t)}{2}}.
        \]
        

    \subsection*{Legendre Transformation}
        Um die Hamiltonfunktion aufstellen zu können, braucht es eine Transformation der Form $p(t):=\dv{b}L(t,(q(t),b))|_{b=q'(t)}$, sodaß wir durch 
        \[
            \dv{b}L(t,(q(t),b))|_{b=q'(t)} = \dv{b}\Biggl(
                \frac12\cdot m_S\cdot\dabs{b}{2} - G\cdot\sum_{i\in[2]}\frac{m_i\cdot m_S}{\dabs{Q_i^*(t) - q(t)}{2}}    
            \Biggr)
            = m_S\cdot q'(t)
        \]
        und Substitution $q'(t) = p(t) / m_S$ den Hamiltonian der Form 
        \[
            H(t,(q(t),p(t))) = \frac{p(t)^2}{2m_S} + V(t,q(t))
        \]
        erhalten. 

    \subsection*{Numerik des Hamiltonsystems}
        Um die Theorie der Numerik auf unser Problem anwenden zu können, müssen wir noch eine \emph{rechte Seite} definieren. Wir finden diese über die \emph{Hamiltonischen Bewegungsgleichungen}, welche durch die Richtungsableitungen in $q(t)$ und $p(t)$ gegeben und in einem Tupel zusammengefasst werden. Hierzu bestimmen wir zunächst
        \begin{align*}
            \dv{a}H(t,(a,p(t))) &= \dv{a}\Biggl(
                \frac{p(t)^2}{2m_S} + V(t,a)
            \Biggr) = \dv{a}V(t,a),\\
            \dv{b}H(t,(q(t),b)) &= \dv{b}\Biggl(
                \frac{b^2}{2m_S} + V(t,q(t))
            \Biggr) = \frac{b}{m_S} + \dv{b}V(t,q(t)).
        \end{align*}    
        An dieser Stelle müssen wir noch einmal arbeiten, um die Ableitung des Potentials in $q(t)$ zu bestimmen, da es sich um eine Verkettung höheren Grades handelt. Definiere $g_i:=\bigl(\dabs{Q_i(t)-a}{2}\bigr)_{a\in\R^3}$, dann ist zunächst
        \[
            \dv{a}V(t,a) = G\cdot\sum_{i\in[2]}\dv{a}\frac{m_i\cdot m_S}{g(a)} = G\cdot\sum_{i\in[2]}m_i\cdot m_S\cdot\dv{a}\frac{1}{g_i(a)},
        \]
        sodaß wir durch Kenntnis der Ableitung von inv auf $\R$, gegeben durch $d$inv$(x)(1) = ($inv$\circ q_2)(x)$ mit $q_2$ als Quadratmonom fernder schreiben können:
        \[
            \dv{a}\frac{1}{g_i(a)} = d(\inv\circ g_i)(a)(1) = (\inv\circ q_2\circ g_i)(a)\cdot dg_i(a)(1) = \frac{1}{g_i(a)^2}\cdot \dv{a}g_i(a).  
        \]
        % Dabei verstecken wir in $h_a = (0,(1,0))$ die verwendete Ableitungsrichtung gemäß der tatsächlichen Argumente von $H$. 
        Ferner ist $dg(x)(h_a)$ zu evaluieren, was wir durch $l_{i}:=\bigl((Q_i(t) - x)^2\bigr)_{x\in\R}$ fortsetzen zu
        \[
            dg_i(a)(h) = \sum_{k\in[3]}dg_i(x)(h_k\cdot\mbbEins_k), \qquad\mbbEins_{k}:=\fdef{\begin{cases}
                1 & i = k\\
                0 & \text{sonst}
            \end{cases}}{i\in[3]}
        \]
        sodaß $g$ für die Richtung $h=1\in\R^3$ die Form 
        \[
            dg_i(a)(\mbbEins_k) = \dv{a_k}\left.\nbra{l_i(a_k) + \sum_{j\in[3]\setminus{\{k\}}}l_{i}(a_j)}^{\frac12}\right|_{a = x_j}
        \]
        besitzt, nach Definition der Quadratnorm. Leiten wir beispielsweise in Richtung $\mbbEins_1$ ab, so erhalten wir für $\tilde L_i(a,k)$ als Zusammenfassung des Wurzelinhaltes
        \[
            \dv{a_1}\sqrt{\tilde L_i(a,1)} = \frac{1}{\sqrt{\tilde L_i(a,1)}}\cdot\dv{a_1}\tilde L_i(a,1) = \frac{1}{\sqrt{\tilde L_i(a,1)}}\cdot 2\cdot (Q_i(t) - a_1). 
        \]
        Dadurch ergibt sich für die gesamte Richtung der Ausdruck 
        \[
            \dv{a}g_i(x)(h_a) = \sum_{k\in[3]}\frac{1}{\tilde L(a,k)}\cdot 2\cdot (Q_i(t) - a_k),
        \]
        wobei \enquote{der Rest der Summe} wegen fehlender $a_k$ Abhängigkeit stets wegfällt und unsere anfangs gesuchte Ableitung des Potentials in $a$ sich zu 
        \[
            \dv{a}V(t,a) = G\cdot\sum_{i\in[2]}m_i\cdot m_S\cdot\frac{1}{g_i(a)}\cdot\sum_{k\in[3]}\frac{1}{\tilde L(a,k)}\cdot 2\cdot (Q_i(t) - a_k)
        \]
        zusammensetzt.
        sowie $\dv{b}V(t,q(t)) = 0$. Damit erhalten wir die Hamiltonischen Bewegungsgleichungen und gleichzeitig durch $F(t,x(t)):=\bigl(-\dv{b}H(t,(x_1^*(t),b)),\dv{a}H(t,(a,x_2^*(t)))\bigr)$ die rechte Seite des Hamiltonsystems:
        \begin{align*}
            F(t,x(t)) = \begin{pmatrix}
                -x_2^*(t)/m_S\\
                \dv{a}V(t,a)|_{a=x_1^* (t)}
            \end{pmatrix}.
        \end{align*}
        So können wir dann die rechte Seite $F=J^T\cdot\nabla H:\R\times\R^{2\cdot 3}\to\R^{2\cdot 3}$ durch die Matrix
        \[
            J:=\begin{pmatrix}
                0 & I_{3} \\
                -I_{3} & 0
            \end{pmatrix}
        \]
        darstellen.



\end{document}